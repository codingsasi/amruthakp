%% start of file `template.tex'.
%% Copyright 2006-2013 Xavier Danaux (xdanaux@gmail.com).
%
% This work may be distributed and/or modified under the
% conditions of the LaTeX Project Public License version 1.3c,
% available at http://www.latex-project.org/lppl/.


\documentclass[11pt,a4paper,sans]{moderncv}        % possible options include font size ('10pt', '11pt' and '12pt'), paper size ('a4paper', 'letterpaper', 'a5paper', 'legalpaper', 'executivepaper' and 'landscape') and font family ('sans' and 'roman')

% moderncv themes
\moderncvstyle{classic}                             % style options are 'casual' (default), 'classic', 'oldstyle' and 'banking'
\moderncvcolor{blue}                               % color options 'blue' (default), 'orange', 'green', 'red', 'purple', 'grey' and 'black'
%\renewcommand{\familydefault}{\sfdefault}         % to set the default font; use '\sfdefault' for the default sans serif font, '\rmdefault' for the default roman one, or any tex font name
%\nopagenumbers{}                                  % uncomment to suppress automatic page numbering for CVs longer than one page

% character encoding
\usepackage[utf8]{inputenc}                     % if you are not using xelatex ou lualatex, replace by the encoding you are using
%\usepackage{CJKutf8}                              % if you need to use CJK to typeset your resume in Chinese, Japanese or Korean

% adjust the page margins
\usepackage[scale=0.8]{geometry}
\setlength{\hintscolumnwidth}{3cm}                % if you want to change the width of the column with the dates
%\setlength{\makecvtitlenamewidth}{10cm}           % for the 'classic' style, if you want to force the width allocated to your name and avoid line breaks. be careful though, the length is normally calculated to avoid any overlap with your personal info; use this at your own typographical risks...

% personal data
\name{Amrutha}{K P}
\title{}                               % optional, remove / comment the line if not wanted
\address{Qrts. No. E/III/10, 7th Cross Road}{Willingdon Island, Cochin - 3}{Kerala, India}% optional, remove / comment the line if not wanted; the "postcode city" and and "country" arguments can be omitted or provided empty
\phone[mobile]{\textsf{(Mumbai)} +91~70457~05016}                    % optional, remove / comment the line if not wanted
\phone[mobile]{\textsf{(Kerala)} +91~96339~22993} 
\email{amruthakp56@gmail.com}                               % optional, remove / comment the line if not wanted
\homepage{www.linkedin.com/in/amruthakp}                         % optional, remove / comment the line if not wanted
%\photo[64pt][0.4pt]{picture}                       % optional, remove / comment the line if not wanted; '64pt' is the height the picture must be resized to, 0.4pt is the thickness of the frame around it (put it to 0pt for no frame) and 'picture' is the name of the picture file
\quote{My objective is to work in a stimulating and challenging environment where I can contribute my skills for the success of the organization while gaining expertise in the same.}                                 % optional, remove / comment the line if not wanted

% to show numerical labels in the bibliography (default is to show no labels); only useful if you make citations in your resume
%\makeatletter
%\renewcommand*{\bibliographyitemlabel}{\@biblabel{\arabic{enumiv}}}
%\makeatother
%\renewcommand*{\bibliographyitemlabel}{[\arabic{enumiv}]}% CONSIDER REPLACING THE ABOVE BY THIS

% bibliography with mutiple entries
%\usepackage{multibib}
%\newcites{book,misc}{{Books},{Others}}
%----------------------------------------------------------------------------------
%            content
%----------------------------------------------------------------------------------
\begin{document}
%\begin{CJK*}{UTF8}{gbsn}                          % to typeset your resume in Chinese using CJK
%-----       resume       ---------------------------------------------------------
\makecvtitle

\section{Skills}
\cvlistitem{Excellent communication skills in English}
\cvlistitem{Technical writing}
\cvlistitem{Creative writing}
\cvlistitem{Reporting}
\cvlistitem{Proofreading and editing}
\cvlistitem{Research and analytical skills}
\cvlistitem{Leadership qualities}
\cvlistitem{Team player}
\cvlistitem{Good working competency in MS Office tools (MS Document, MS Excel, MS Powerpoint)}
\cvlistitem{Film making and post production skills : Has good working competency in \textbf{Final Cut Pro} and \textbf{Soundtrack Pro}.}
\cvlistitem{Photography and post processing : Experience working with DSLR and digital cameras and has good work competency in \textbf{Adobe Photoshop}.\newline{}}

\section{Experience}
\cventry{April -- May 2016}{Intern}{The Hindu}{Thiruvananthapuram}{}{Worked as a local beat reporter for six weeks. My work included visiting venues of events, attending seminars and press conferences, and researching to compile information and prepare news reports. Produced six stories with by-line and several other reports on local events.\newline{}
\href{http://www.thehindu.com/profile/author/amrutha-k.p}{\color{color1}{thehindu.com/profile/author/amrutha-k.p}} - Link to The Hindu website \newline{}}

\cventry{December 2014}{Compere}{19th International Film Festival of Kerala (IFFK)}{Thiruvananthapuram}{}{Presented information about each film before its screening; prepared and edited scripts for compering. \newline{}}

\cventry{November 2014}{Media Coordination}{22nd DC International Book Fair and Science Festival}{Thiruvananthapuram}{}{ Coordinated student reporters; proofread and edited press releases compiled by them. I was also the \textbf{sub-editor} of three special editions of '\textbf{Netra}' (a newspaper journal published by the Department of Journalism, Mass Communication and Video Production, Mar Ivanios College, Thiruvananthapuram), published as part of the festival.\newline{}}

\cventry{}{Editor}{\textbf{English Speeches: Vol 4; ISBN 93-83686-32-4}}{}{}{Published and distributed by Vimala Books, Kanjirapally; First Edition: October 2014. \newline{}}

\cventry{January 2014}{Media cell member}{International Conference on Deepening Democracy - 2014}{Kovalam}{Thiruvananthapuram}{Organized by Department of Local Self Government, Government of Kerala. \newline{}
Was part of a team that prepared and circulated press releases, conducted interviews and compiled new reports.\newline{}}

\cventry{}{Co-author}{\textbf{English Speeches: Vol 3; ISBN 93-81504-90-3}}{}{}{Published and distributed by Vimala Books, Kanjirapally; First Edition: October 2013 \newline{}
The experience gave me insight into the process of book publishing. In a team of three, did research, proofreading, editing and content writing.\newline{}}

\cventry{February 2013}{Student Reporter}{Scientia - 2013}{Thiruvananthapuram}{}{A national seminar on the relevance of GM crops in Food security.\newline{}
Compiled reports and press release for publication by leading newspapers.\newline{}
\newline{}}

\cventry{December 2012}{Compere}{17th International Film Festival of Kerala (IFFK)}{Thiruvananthapuram}{}{Presented information about each film before its screening; prepared and edited scripts for compering. \newline{}
\newline{}
Worked as a reporter in several intra-collegiate programmes and national seminars and prepared press releases.\newline{}}

\section{Academic Projects}
\cvlistitem{M.A. dissertation titled \textbf{\textit{Kiss of Love: A Study exploring Protest, Spaces and Manifestations.\newline{}}}The study attempts to explore how social media was used to organize Kiss of Love, a protests that employed digital media, from the perspective of context and meanings. The project tries to develop an understanding of meanings that individual participants attributed to the protest, the role social media plays and the themes and meanings that emerged out of the conversations and discussions that happened on the event's Facebook page.\newline{}}

\cvlistitem{\textbf{\textit{Wheels of Her Own} - Documentary.}\newline{}
Was involved in all aspects of production, from research and scripting to camera, sound and direction of this 20 minute documentary. Created as part of the final media project in a group of five members, the film explores lives of two women who have their own wheels, coming from different strata of society. The film questions the kind of mobility these women have and the notions of empowerment that get attached to the concept of owning one's own rides in the city. The work highlights these women's negotiation with the daily life, the family and work along with the understated motif of difference between work and leisure.\newline{}}

\cvlistitem{\textbf{\textit{These Women in the Hills} - Short Documentary on waste workers.}\newline{}
Was involved in all aspects of production, from research and scripting to camera, sound and direction of this 25 minute documentary. Produced as part of the course ’Working with Video II’ in a team of four, the film tries to look at the waste workers at the Deonar Dumping ground and the kind of troubles they go through during their work from a health and safety perspective, questioning the popular middle class notion of ‘it being dirty’. The film implicates the city in the process of looking at the way waste work is treated within such a space and explores the general dynamics within the Deonar Dumping Ground, of who is allowed and who is not allowed. It broadly tells the story of the situation within the dumping ground, following the fire, which left families devastated not only with illness but also with the prospect of losing one’s only means of livelihood. \href{https://www.youtube.com/watch?v=SVlHWdx6YVQ&t}{\color{color1}{'These Women in the Hills' YouTube link}}\newline{}}

\cvlistitem{\textbf{\textit{The Nation Wants to Know} - Public Service Advertisement on censorship.}\newline{}
Was involved in all aspects of production, from research and scripting to camera, sound and direction of this 2.35 minute long PSA. Produced as part of the course 'Working with Video I' in a team of four, the PSA broadly explores censorship of various forms imposed by mainstream media corporation on discussions related to specific topics such as caste, religion, LGBTQAI+ rights and beef ban. \href{https://www.youtube.com/watch?v=C-YA6xfXS0w&t}{\color{color1}{'The Nation Wants to Know' YouTube link}}\newline{}}

\cvlistitem{\textbf{\textit{Dil Khoon Ke Hamaare} - Music Video.}\newline{}
Was involved in all aspects of production, from research and scripting to camera, sound and direction of this 6.45 minute long music video. Produced as part of the course 'Working with Video I' in a team of four, the video portrays the lives of two protagonists—a queer college lecturer and a dancer who is in conflict about her sexuality—in the context of legal struggles around Section 377. \href{https://www.youtube.com/watch?v=ZME9_as1mSI}{\color{color1}{'Dil Khoon Ke Hamaare' YouTube link}}\newline{}}

\cvlistitem{B.A. dissertation titled \textbf{\textit{Digital Literacy and Age Barrier: Senior citizens and their position in the e-World}} as part of course requirement for Bachelor's Degree.\newline{}
The study done in a group of six, explores the extent of digital literacy, and the nature and pattern of internet use among senior citizens in Thiruvananthapuram, Kerala in the backdrop of increasing number of e-governance initiatives such as Akshaya centres, 'FRIENDS' or Fast Reliable Instant Effective Network for Disbursement of Service,  National Digital Literacy Mission etc. While also exploring if the use of internet is helpful in dealing with social alienation during old age, the dissertation attempted to add insight into the often ignored area of digital literacy, marginalisation and old age.\newline{}}

\cvlistitem{\textbf{Sub-editor, reporter and layout artist} for several editions of '\textbf{Netra'}, newspaper journal published by the Department of Journalism, Mass Communication and Video production, Mar Ivanios College, Thiruvananthapuram.\newline{}}

\cvlistitem{Worked as part of a team to script and produce a short film, '\textbf{\textit{A Sign of Hope}}'.\newline{}}

\cvlistitem{Co-scripted and produced a Public Service Advertisement on fast food.\newline{}}

\cvlistitem{Participated in several media seminars and conferences.\newline{}\newline{}}

\section{Education}
\cvitemwithcomment{2015 -- 2017}{\textbf{M.A. Media and Cultural Studies}}{GPA 7.6/10 (S3)}
\cvitemwithcomment{}{School of Media and Cultural Studies}{GPA 7.5/10 (S2)}
\cvitemwithcomment{}{Tata Institute of Social Sciences, Mumbai}{GPA 7.7/10 (S1)}
\cvitemwithcomment{}{}{}

\cvitemwithcomment{2012 -- 2015}{\textbf{B.A. Journalism, Mass Communication and Video Production}}{CGPA 3.72/4}
\cvitemwithcomment{}{Mar Ivanios College, Thiruvananthapuram}{2nd Rank,}
\cvitemwithcomment{}{University of Kerala}{University of Kerala}
\cvitemwithcomment{}{}{}

\cvitemwithcomment{2000 -- 2012}{\textbf{I-XII}}{}
\cvitemwithcomment{}{Kendriya Vidyalaya Cochin Port Trust, W/Island, Kochi, Kerala}{}
\cvitemwithcomment{}{CBSE}{}
\cvitemwithcomment{}{}{}

\section{Co-curricular activities, awards and recognition}

\cvlistitem{Has actively participated in competitions in 'Quintissence', the cultural fest of TISS, Mumbai}

\cvlistitem{'\textbf{Best Student} in Final Year Journalism, Mass Communication and Video Production' in the year 2014 - 2015}

\cvlistitem{Event head of 'The Best Journalist', a competition held as a part of 'Ivano Fest', the inter-collegiate cultural fest of Mar Ivanios College.}

\cvlistitem{Won miscellaneous prices for painting, calligraphy, film quiz and other literary competitions at school and college levels as well as inter-collegiate fests.\newline{}}

\section{Interests}
\cvlistdoubleitem{Reading}{Writing}
\cvlistdoubleitem{Research}{Dabbling in popular Science}
\cvlistdoubleitem{Painting}{Photography}

\section{Languages}
\cvitemwithcomment{}{English}{Full professional proficiency}
\cvitemwithcomment{}{Malayalam}{Mother tongue}
\cvitemwithcomment{}{Hindi}{Average proficiency}


% Publications from a BibTeX file without multibib
%  for numerical labels: \renewcommand{\bibliographyitemlabel}{\@biblabel{\arabic{enumiv}}}% CONSIDER MERGING WITH PREAMBLE PART
%  to redefine the heading string ("Publications"): \renewcommand{\refname}{Articles}

                        % 'publications' is the name of a BibTeX file

% Publications from a BibTeX file using the multibib package
%\section{Publications}
%\nocitebook{book1,book2}
%\bibliographystylebook{plain}
%\bibliographybook{publications}                   % 'publications' is the name of a BibTeX file
%\nocitemisc{misc1,misc2,misc3}
%\bibliographystylemisc{plain}
%\bibliographymisc{publications}                   % 'publications' is the name of a BibTeX file

\end{document}


%% end of file `template.tex'.
